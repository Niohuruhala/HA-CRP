\chapter{拓扑性质}
本章主要兼顾复习和延拓,着眼于:几个常用的拓扑性质(分离性、可数性、紧致性和连通性)、一些对拓扑公理的补充(\(T_i,C_j \text{公理},\quad i=1,2,3,4 ; \quad j=1,2 \))以及一些重要定理。
\section{分离公理和可数公理}
分离公理都是关于两个点(或者是闭集)能否用领域来分隔的性质,是对拓扑空间的附加要求.
\begin{definition}
    \(T_1\) \textbf{公理}:任何两个不同点x与y,x有邻域不含y,y有邻域不含x . 
\end{definition}
\begin{definition}
    \(T_2\) \textbf{公理} : 任何两个不同点有不相交的邻域 .
\end{definition}
\begin{definition}
     \(T_3\) \textbf{公理} :任何一个点都与不含它的闭集有不相交的(开)邻域
\end{definition}
\begin{definition}
     \(T_4\) \textbf{公理} :任意两个不相交的闭集有不相交的(开)邻域
\end{definition}
\begin{note}
    在上述分离公理中: 邻域改变为“开邻域”上述公理的意义不变.
\end{note}
显然的是,如果一个拓扑空间满足\(T_2\)公理,那么它一定满足\(T_1\)公理 . 但是\(T_1\)公理不能推得\(T_2\)公理,这样的话,能够揭示:\(T_1\)公理是一个比\(T_2\)公理弱的公理.
依照点与点之间的公理和点(集合)与集合的分类介绍分离公理:
\subsection*{\(T_1\)公理与\(T_2\)公理}
\begin{corollary}\label{T1 公理 有限 闭集}
    X是一个拓扑空间,那么: X满足\(T_1\)公理 \(\Leftrightarrow\) X的有限子集是闭集.
\end{corollary}
\begin{proof}
    先证明充分性: 
    若X满足\(T_1\)公理,那么:对于X中的任意两点x,y,有x的邻域不含y.那么意味着\(\left\{x\right\}\)是闭集(因为:\(\left\{x\right\}\)的余集是开集) .
    证明必要性:
    若X的有限子集是闭集,那么:取X中的一个有限子集A,那么对于\(A^{c}\)是开集,所以对于其任意一点均是它的内点.因此对于\(\forall x \in X \quad \left\{x\right\}\)来说: 对于X中的任意两点x,y,有x的邻域不含y .因此必要性得证
\end{proof}
\begin{lemma}
    若X满足\(T_1\)公理,\(A \subset X\),点x是A的聚点,则x的任意邻域与A的交是无穷集.
\end{lemma}
\begin{proof}
    反证法: 若x的任意邻域与A的交是有限集U,\(U \cap A\)是有限集,不妨设U是开集,记\(B=(U\cap A ) \ \left\{x\right\}\),它是有限集,因此由于命题\ref{T1 公理 有限 闭集}可知,它是闭集.因此\(U \cap B^{c}\)仍是x的开邻域,矛盾! 
\end{proof}
\begin{corollary}
    若X满足\(T_2\)公理,一个序列不会收敛到两个以上的点.
\end{corollary}
\begin{proof}
    反证法: 若一个序列\(\left\{x_n\right\}\)收敛到两个点\(x_1,x_2\),由于X满足\(T_2\)公理,那么可以取\(x_1,x_2\)的两个不相交的邻域 U和V . 因此根据收敛序列的定义可知,U包含了几乎所有的\(\left\{x_n\right\}\)的点,那么则意味着V只能包含有限个\(\left\{x_n\right\}\)的点,矛盾!
\end{proof}
上述命题揭示了\(T_2\)公理的重要性,它能够保证序列收敛的唯一性,因此人们常常成满足\(T_2\)公理的拓扑空间为\(\textbf{Hausdorff空间}\),但是声明一点:\(T_2\)公理的作用不只是改善序列收敛性,它还有其余的作用.
\subsection*{\(T_3\)公理与\(T_4\)公理}
如果一个拓扑空间满足\(T_1\)公理和\(T_3\)公理,那么它也满足\(T_2\)公理,这是因为若X满足\(T_1\)公理,那么它的单点集是闭集,那么利用\(T_3\)公理可知:X满足\(T_2\)公理 . 
\begin{corollary}
    度量空间(X,d)满足\(T_i\)公理(i=1,2,3,4)
\end{corollary}
\begin{proof}
    易知:度量空间中单点集一定是闭集,所以度量空间满足\(T_1\)公理,只需验证其满足\(T_4\)公理,这是因为(\(T_4\)公理和\(T_1\)公理可以推得所有分离公理.下面证明度量空间满足\(T_4\)公理.引用一连续函数\begin{align}
        f(x)=\frac{d(x,A}{d(x,A)+d(x,B)}
    \end{align}
    其中A,B是不相交闭集,不妨设它们都不是\(\emptyset\) 不失一般性,因此取\(t \in (0,1)\) 则\(f^{-1}((-\infty,t)),f^{-1}((t,+\infty))\)分别是A和B的不相交邻域.
\end{proof}
\begin{corollary}\label{T_3 T_4 证明定理}
    \begin{enumerate}
        \item 满足\(T_3\)公理 \(\Leftrightarrow\)任意点x和它的开邻域W,存在x的开邻域U,使得\(\overline{U} \subset W\) \\
        \item 满足\(T_4\)公理 \(\Leftrightarrow\) 任意闭集A和它的开邻域W ,存在A的开邻域U 使得\(\overline{U} \subset W\)
    \end{enumerate}
\end{corollary}
\begin{proof}
    证明略(绝大多数拓扑学教科书都有证明)
\end{proof}
\begin{note}
    上述命题\ref{T_3 T_4 证明定理}是十分重要的定理,有时由于定义的繁琐,我们不方便证明,大多数情况可以用它来代替.
\end{note}
\begin{definition}
    我们称满足\(T_4(T_3)\)公理的空间成为正则空间.
\end{definition}
\subsection*{可数公理}
可数公理有两个,分别称为第一可数公理和第二可数公理,简称为\(C_1\)公理和\(C_2\)公理.
为了能够定义\(C_1\)公理,首先给出邻域基的定义:
\begin{definition}
    设\(x \in X\) 把x的所有邻域的集合称为x的邻域系,记作\(\mathscr{N} (x)\) ,\(\mathscr{N} (x)\) 的一个子集(即x的一族邻域) \(\mathscr{U}\)称为x的一个邻域基,如果x的每个邻域至少包括\(\mathscr{U}\)中的一个成员.
\end{definition}
\begin{example}
    \(\mathscr{N}(x)\)本身就是x的一个邻域基.x的所有开邻域是x的一个邻域基.
\end{example}
\begin{definition}
    \(C_1\mathbf{公理}\) 任一点都有可数的邻域基
\end{definition}
容易证明:度量空间是\(C_1\)空间.(因为\(\left\{B(x,q)| q\text{是正有理数}\right\}\)构成了\(\forall x \in \text{度量空间}\)的一个度量空间)
\begin{corollary}\label{邻域基 V_n}
    如果X在x处有可数邻域基,则x有可数邻域基\(\left\{V_n\right\}\),使得\(m > n \) 时 \(V_m \subset V_n\)
\end{corollary}
\begin{proof}
    取x的一个邻域基\(\left\{U_i\right\}\),进而记\(V_n =\bigcap_{i=1}^{n}U_n\),因此 \(V_n\)也是一个邻域基,所以当\(n>m\)时,我们有 \(V_n \subset V_m\).
\end{proof}
\begin{corollary}\label{C_1收敛推论}
    若X是\(C_1\)空间,\(A\subset X,x \in \overline{A}\),则A中存在收敛到x的序列.
\end{corollary}
\begin{proof}
    取x的一个邻域基\(\left\{V_n\right\}\),由于\(x \in \overline{A}\),所以 \(\overline{A} \cap V_n \neq \emptyset \)进而\(A \cap V_n \neq \emptyset\),取\(x_n \in A \cap V_n\)因此根据命题\ref{邻域基 V_n},可知 \(m >n \quad V_m \subset V_n\) ,可以取得一个序列\(\left\{x_n\right\}\)并且这个序列收敛到x
\end{proof}
\begin{lemma}
    若X是\(C_1\)空间,\(x_0 \in X\),映射\(f: X \rightarrow Y \)满足: 当\(x_n \rightarrow x_0\)时,\(f_n(x)  \rightarrow f_0(x)\),则\(f\)在\(x_0\)处连续
\end{lemma}
\begin{proof}
    反证法: 如果f在\(x_0\)不连续,则存在\(f(x_0)\)的邻域V 使得\(f^{-1}(V)\)不是\(x_0\)的邻域.即\(x_0 \in \overline{(f^{-1}(V))^{c}}\) ,根据命题\ref{C_1收敛推论}可知:在\((f^{-1}(V))^{c}\)中存在收敛到\(x_0\)的序列,即\(f(x_n) \rightarrow f(x_0)\) ,因此对于所有\(x_n\)来说, \(x_n \in V \)但这与题设矛盾,因此命题成立.
\end{proof}
\begin{definition}
    \(C_2 \textbf{公理}\) 拓扑空间有可数拓扑基
\end{definition}
需要注意的是:有些度量空间也不是\(C_2\)空间,例如 在R中,规定度量d(离散度量)为\begin{align}
    d(x,y)= \begin{cases}
        0   \qquad &x=y \\
        1   \qquad &x\neq y 
    \end{cases}
\end{align} 
 则(R,d)是离散度量空间,任何一点都是开集,于是对于它的任意拓扑基都要包括所有的单点集.显然:它的拓扑基不可数故而它一定不是\(C_2\)空间.
\begin{corollary}
    可分度量空间一定是\(C_2\)空间
\end{corollary}
\begin{example}
    \(\textbf{希尔伯特空间}\)\(E^{\omega}\)是一个度量空间,在所有平方收敛的实数序列构成的线性空间中,规定内积
    \[(\left\{x_n\right\},\left\{y_n\right\})= \sum\limits_{n=1}^{\infty}x_ny_n\]
    它决定度量\(\rho\):
    \[\rho(\left\{x_n\right\},\left\{y_n\right\}) = \sqrt{\sum\limits_{n=1}^{\infty}(x_n-y_n)^2}\]
    得到的度量空间就是\(E^{\omega}\)
\end{example}
\begin{theorem}[Lindelof 定理]\label{林登夫定理}
    若拓扑空间满足\(C_2 ,T_3\)公理,则它也满足\(T_4\)公理
\end{theorem}
\subsection*{拓扑空间的遗传性与可乘性}
对于一个拓扑性质,如果一个拓扑空间具有它时,子空间也必具有它,则这种拓扑性质称为具有遗传性;如果两个空间都具有它时,它们的乘积空间也具有它,则这种拓扑性质称为具有可乘性. 
\\
可分性是可乘的,但是不是遗传的;对于分离定理: \(T_1,T_2\)和\(T_3\)公理都具有可乘性和遗传性,但是\(T_4\)公理不具有这两种性质.
\subsection*{重要定理}
\begin{definition}
    如果拓扑空间X中任意两点,必有一开集只包含其中一点,则称X满足\(T_0\)公理
\end{definition}
\begin{theorem}
    如果X满足\(T_0\)公理和\(T_3\)公理,则它也满足\(T_2\)公理
\end{theorem}
\begin{theorem}
    设X满足\(T_1\)公理,则X中任一子集的导集是闭集
\end{theorem}
\begin{theorem}
    Hausdorff空间的子空间也是Hausdorff空间.
\end{theorem}
\begin{theorem}
    两个Hausdorff空间的乘积空间也是Hausdorff空间
\end{theorem}
\begin{theorem}
    设X满足\(T_3\)公理,则F为X的闭集,\(x \notin F\) ,则 存在F和x的开邻域U和V,使得\(\overline{U}\cap \overline{V} = \emptyset \)
\end{theorem}
\begin{theorem}
    设\(f: X \rightarrow Y \)是满的闭连续映射,若X满足\(T_4\)公理,则Y满足\(T_4\)公理
\end{theorem}
\begin{theorem}
    如果X是\(C_1\)空间,并且它的序列最多只能收敛到一个点,那么X是Hausdorff空间
\end{theorem}
\begin{theorem}
    可分度量空间的子空间也是可分空间
\end{theorem}
\section{Uryshon引理及其应用}
\begin{theorem}[Uryshon引理]
    X是一个满足\(T_4\)公理的拓扑空间,那么对于X的两个不相交闭集A和B,存在X上的连续函数f,它在A和B上分别取值为0和1.
\end{theorem}
对于Uryshon定理的证明不作任何要求(事实上:作者自己也写不出来完整的证明过程,只需要了解证明过程的技术即可: 利用构造开邻域)
\begin{proof}
    记\(Q_I\)是\([ 0 ,1]\)中的有理数的集合,它是一个可数集,证明分两步:
    (1)
    \begin{enumerate}
        \item 用递归定义,构造开集族:\({ U_r : r \in Q_t } \),使得 \\
        \item 当\(r < r^{'}\)时,\(\overline{U_r}\subset U_{r^{'}}\) \\
        \item \(\forall Q_I , A \subset U_r \subset B^{c}\)
    \end{enumerate}
    作法如下,将\(Q_I\)随意地排列为 \({ r_1 r_2 ,\dots }\),只须令\(r_1 = 1 \quad r_2 = 0\)然后对n归纳地构造\(U_{r_n}\),取\(U_{r_1} = B^{c}\),它是A的开邻域,因此可以构造\(U_{r_2}\)
    是A的开邻域,\(\overline{U_{r_2}} \subset U_{r_1}\) \\
    设\(U_{r_1},U_{r_2} \dots U_{r_n}\)已构造,它们满足(1)和(2) , 记\(r_{i(n)}= \max { r_l | l \leq n r_l < r_{n+1} } \quad r_{j(n)}= \min  { r_l | l \leq n r_l > r_{n+1} } \),则 \(r_{i(n)} <r_{j(n)} \) 因此 \(\overline{U_{r_{i(n)}}} \subset U_{r_{i(n)}}\).从而我们得到了\({ U_r }\)的定义完成.
    (2) 规定函数\(f: X \rightarrow E^1\)为 \(\forall x \in X \) :
    \[f(x) = \sup \left\{r \in Q_I | x \notin U_r \right\} = \inf \left\{ r \in Q_I | x \in U_r \right\}\]
    这里给出f(x)的两个定义式,如果\(\forall r ,x \notin U_r\)  则用第一式,如果\(\forall r,x \in U_r\),则用第二式;余下的情形,两式是相等的。因为\(A \subset U_r , \forall r \in Q_I\),所以f在A上各点的值都为0;类似地,f在B上各点取值1.现在只剩下f连续性的验证。显然对于任何开区间\((a,b)\),\(f^{-1}(a,b)\)是X的开集,即它的每一点都是内点。故而f在X中连续。
    \end{proof}
    显然,当对于A,B有定理所述的连续函数时,A,B有不相交的(开)邻域。因此Uryshon引理是\(T_4\)公理的等价命题。
    进而拓扑学家们还发现了一个更加深入的等价于Uryshon引理的定理:\(\textbf{Tietze扩张定理}\)  
    \begin{theorem}[Tietze扩张定理]
    如果拓扑空间X满足\(T_4\)公理,则定义在X的闭子集F上的连续函数可以连续扩张到X上。
    \end{theorem}
    可以看出:Tietze扩张定理也是\(T_4\)公理的一个等价命题。
    拓扑空间\((X ,\tau )\)称作可度量化的,若可以在集合X上找到一个度量d,使得\(\tau_d = \tau\).
    \begin{corollary}
        拓扑空间X可度量化\(\Leftrightarrow\)存在从X到一个度量空间的嵌入映射.
    \end{corollary}
    \begin{theorem}[Uryshon度量化定理]
        拓扑空间X如果满足\(T_1,T_4,C_2\)公理,则X可以嵌入到希尔伯特空间\(E^{\omega}\)中
    \end{theorem}
    Uryshon度量化定理给出了判断拓扑空间是否是可度量的一种新方法,它使得人们更加方便地处理度量化问题(相较于定义证明而言)。
    \begin{note}
        事实上,如果一个拓扑空间满足\(T_1,T_4,C_2\)公理,那么则意味着它满足了所有的分离公理和可数公理(请读者自己证明)。
    \end{note}
    \subsection*{补充定义}
    \begin{definition}
        拓扑空间Y的子集B称为Y的一个收缩核,如果存在连续映射\(r: Y \rightarrow B\),使得\(\forall x \in B \quad r(X) = x\);称r为Y到B的一个收缩映射。
    \end{definition}
    \begin{theorem}
        设D是\(E^n\)的收缩核,X满足\(T_4\)公理,A是X的闭集,则存在连续映射\(f : A \rightarrow D\)可以扩张到X上 .
    \end{theorem}
    \section{紧致性}
    紧致性在分析学中早就出现并有许多应用,然而从本质上讲,它是属千拓扑学范畴的概念,并且是一种最基本、最常见的拓扑性质
    \subsection*{紧致和列紧}
    在分析学中紧致性(在那里它等价千列紧性)早就显示了它的威力.有界闭区间上的连续函数是有界的,达到它的最大、最小值,并且是一致连续的.在证明这些结论时都用到了同一事实:有界闭区间上的每个序列有收敛的子序列.这种性质后来称为“列紧性”(自列紧),它可以一字不改地推广到一般拓扑空间中.
    \begin{definition}
        拓扑空间称为列紧的,如果它的每个序列有收敛(即有极限点)的子序列.
    \end{definition}
    \begin{corollary}
        定义在列紧拓扑空间X上的连续函数\(f: X \rightarrow E^1\)有界,并达到最大,最小值
    \end{corollary}
    在给出紧致性的定义前,先回顾以下我们对覆盖和子覆盖的定义,它是X的一个子集族\(\mathscr{U}\),满足\(\bigcup_{U \in \mathscr{U}}\),则称\(\mathscr{U}\)是集合X的一个覆盖.如果覆盖 \(\mathscr{U}\) 中只含有有限个子集,就称\(\mathscr{U}\) 有限覆盖。如果\(\mathscr{U}\) 的一个子集族\(\mathscr{U}^{'} \subset \mathscr{U} \) 本身也构成X的覆盖,就称\(\mathscr{U}^{'}\)是\(\mathscr{U}\)的子覆盖.
    \begin{definition}
        拓扑空间称为紧致的,如果它的每个开覆盖有有限个子覆盖
    \end{definition}
    在度量空间中紧致和列紧是等价的.
    \subsection*{度量空间}
    \begin{corollary}
        紧致\(C_1\)空间是列紧的
    \end{corollary}
    引入\(\delta-\text{网}\)的概念
    \begin{definition}
        度量空间(X,d)的子集A称为X的一个\(\delta-\text{网}\),如果\(\forall x \in X , d(x,A) < \delta\),即\(\bigcup_{a \in A }B (a,\delta) = X \)
    \end{definition}
   \begin{corollary}\label{网}
       对于任给\(\delta > 0\),列紧度量空间存在有限的\(\delta- \text{网}\)
   \end{corollary}
    根据命题\ref{网}可知: 所有列紧空间都是有界的\\
    设\(\mathscr{U}\)是列紧度量空间\((X,d)\)的一个开覆盖,并且\(x \notin \mathscr{U}\),规定X上函数\(\varphi_{\mathscr{U}}: X \rightarrow E^{1}\)为\[\varphi_{\mathscr{U}}(X) := \sup{ d(x,U^{c} | U \in \mathscr{U} } , \quad \forall x \in X \]
    因为X是有界的,有M,使得\(d(x,y) \leq M ,\quad \forall x ,y \in X \),所以当\(U \neq X\)时,\(d(X,U^{c} ) \leq M \),从而\(\varphi_{\mathscr{U}}(X)\)有意义,又由于\(\mathscr{U}\)是开覆盖,存在\(U \in \mathscr{U}\),使得\(X \in U \),从而\(\varphi_{\mathscr{U}}(X) \geq d(x U^{c}) > 0 \)
    并且通过三角不等式易知\(\varphi_{\mathscr{U}}(X) \leq d(x,y) + \varphi_{\mathscr{U}}(y)\),因此\(\varphi_{\mathscr{U}}(X)\)连续
    \begin{definition}
        设\(\mathscr{U}\)是列紧度量空间(X,d) 的一个开覆盖,\(x \notin \mathscr{U}\),称函数\(\varphi_{\mathscr{U}}\)的最小值为\(\mathscr{U}\)的\(\textbf{Lebesgue数}\), 记作\(L(\mathscr{U})\)
    \end{definition}
    \begin{corollary}
        \(L(\mathscr{U}\);并且当\( 0 < \delta < L(\mathscr{U})\)时,\(\forall x \in X , B(x,\delta)\)必包含在\(\mathscr{U}\)的某个开集U中。
    \end{corollary}
    \begin{proof}
        因为X列紧,所以\(\varphi_{\mathscr{U}}(X)\)在某点\(x_0\)处达到最小值,即\(L(\mathscr{U}) = \varphi_{\mathscr{U}}(X_0) > 0 \) \(\forall x \in X , \delta < L(\mathscr{U}) \leq \varphi_{\mathscr{U}}(X) \),因此存在\(U \in \mathscr{U} \),使得\(d(x,U^{c}) > \delta\),从而\(B(x,\delta) \subset U\)
    \end{proof}
    \begin{corollary}
        列紧度量空间是紧致的 
    \end{corollary}
    \begin{proof}
        设(X,d) 是列紧度量空间,要对它的开覆盖\(\mathscr{U}\)找出有限子覆盖,不妨设\(\mathscr{U}\)之中不包含X,从而又Lebesgue数\(L(\mathscr{U})\) ,取\(\delta < L(\mathscr{U}) \),令\(A = \left\{ a_1 ,a_2 ,\dots a_n \right\} \)是X的\(\delta-\text{网}\).于是\(\bigcup_{i=1}^n B(a_i, \delta )=X \)因此可以有X的有限子覆盖\(\left\{ U_1 ,U_2 , U_n\right\} \quad B(a_i, \delta ) \subset U_i \)
    \end{proof}
    \begin{theorem}
        在度量空间X中 X是列紧\(\Leftrightarrow \) X是紧致 .
    \end{theorem}
    \subsection*{紧致空间的性质}
    “一个拓扑空间X的子集A如果作为子空间是紧致的,就称为X的紧致子集.这里在概念上并没有提出任何新思想.下面介绍判断一个子集是否紧致的办法,实用中它常常比定义方便些.” X中一个开集族\(\mathscr{U}\),如果满足\(A \subset \bigcup_{U \in \mathscr{U}}U \),则称\(\mathscr{U}\)是A在X中的一个开覆盖(区别于A的开覆盖,后者由A中的开集构成)
    \begin{corollary}
        A是X的紧致子集\(\Leftrightarrow\)A在X的任一开覆盖有有限子覆盖
    \end{corollary}
    \begin{proof}
        \(\Leftarrow\): 若A在X的任一开覆盖有有限子覆盖,那么由定义可知:\(\forall V \in \mathscr{U} \),取定X中的开集U,使得\(V = U \cap A\),得到的U构成A在中的开覆盖\(\mathscr{U}\),由条件,可证明A是紧致的.
        \\
        \(\Rightarrow\) : 若A是X的紧致子集,则设\(\mathscr{U}\)是A在X中的开覆盖,则\(\mathscr{U}_A = \left\{U \cap A | U \in \mathscr{U} \right\}\)是A的开覆盖,因为A是紧致的,所以\(\mathscr{U}_A\)有有限子覆盖,所以A在X的任一开覆盖有有限子覆盖
    \end{proof}
    \begin{corollary}
        紧致空间的闭子集紧致
    \end{corollary}
    \begin{proof}
        设X是紧致空间,那么对于X中的闭子集A,我们只需证明:A在X的任一开覆盖有有限子覆盖即可: 因为A是闭集,那么、\(A^{c}\)是开集,从而可令A的开覆盖\(\mathscr{U}\)添加\(A^{c}\)后有X的开覆盖,又由于X是紧致空间,所以可得A在X的任一开覆盖有有限子覆盖.因此紧致空间的闭子集是紧致的
    \end{proof}
    \begin{corollary}
        紧致空间在连续映射下的像也紧致
    \end{corollary}
    \begin{proof}
        设X紧致,并且映射\(f: X \rightarrow Y \)连续,则不妨设\(\mathscr{U}\)是\(f(x)\)在Y中的开覆盖,则\(\left\{ f^{-1}(U) | U \in \mathscr{U}\right\}\)是X的开覆盖.那么有子覆盖\(\left\{f^{-1}(U_1),f^{-1}(U_2),\dots , f^{-1}(U_n)\right\}\),即\(X = \bigcup_{i=1}^n f^{-1}(U_i)\),于是\(f(X)=\bigcup_{i=1}^{n}f(f^{-1}(U_i)) \subset \bigcup_{i=1}^n U_i\).因此f(X)紧致.
    \end{proof}
    \begin{lemma}
        定义在紧致空间上的连续函数有界,并且达到最大、最小值
    \end{lemma}
    \subsection*{Hausdorff空间的紧致子集}
    \begin{corollary}
        若A是Hausdorff空间X的紧致子集,\(X \notin A\),则x与A有不相交的邻域
    \end{corollary}
    \begin{proof}
        \(\forall y \in A \),则\(x \neq y\),X是Hausdorff空间,因而x和y有不相交的开邻域U和V,又由于X是紧致空间,那么U和V分别导出两个有限开覆盖\(\left\{U_i\right\} \quad \left\{V_j\right\}\),同时由于\(U \cap  V_j \subset U \cap V = \emptyset\)
    \end{proof}
    \begin{lemma}
        Hausdorff空间的紧致子集是闭集
    \end{lemma}
    \begin{theorem}
        设\(f; X \rightarrow Y \)是连续的一一对应,其中X紧致,Y是Hausdorff空间,则f是同胚.
    \end{theorem}
    \begin{proof}
        只需证明:\(f^{-1} : Y \rightarrow X\)是连续的,设A是X的闭集,从而A是紧致的,从而f(A)是Y的紧致子集,从而f(A)是Y的闭集,因此\(f^{-1}\)是闭映射,故而\(f^{-1} : Y \rightarrow X\)是连续的
    \end{proof}
    \begin{corollary}
        Hausdorff空间的不相交紧致子集有不相交的邻域
    \end{corollary}
    \begin{corollary}
        紧致Hausdorff空间满足\(T_3 , T_4\)公理
    \end{corollary}
    \subsection*{乘积空间的紧致性}
    注意:紧致性没有遗传性,但它有可乘性.
    \begin{lemma}
        设A是X的紧致子集,y是Y的一点,在乘积空间\(X \times Y \)中,W 是 \(A \times \left\{y\right\}\)的邻域,则存在A和y的开邻域U和V,使得\(U \times V \subset W \)
    \end{lemma}
    \begin{theorem}
        若X和Y均是紧致空间,那么\(X \times Y \)是紧致空间
    \end{theorem}
    \begin{theorem}[吉洪诺夫定理]
        如果规定\(\prod\limits_{\lambda\in \Lambda} X_{\lambda}\)是乘积拓扑,则当每个\(X_{\lambda}\)都是紧致空间时,\(\prod\limits_{\lambda\in \Lambda} X_{\lambda}\)也是紧致空间.
    \end{theorem}
    \subsection*{局部紧致与仿紧}
    “紧致性是一种很好的拓扑性质,但它毕竟太强了,连欧氏空间\(E^n\)也不是紧致的.现在介绍紧致性的两种推广:局部紧致和仿紧.它们在拓扑学以及微分几何等学科中都是较常用到的.但在本书中它们用得很少或不用,这里只介绍它们的定义和最基本的性质.”
    \begin{definition}
        拓扑空间X称为局部紧致的,如果\(\forall x \in X \)都有紧致的邻域
    \end{definition}
    显然的是:紧致空间是局部紧致的,欧氏空间\(E^n\)也是局部紧致的 
    \begin{corollary}
        设X是局部紧致的Hausdorff空间,则
        \begin{enumerate}
            \item X满足\(T_3\)公理; \\
            \item \(\forall x \in X \) ,x 的紧致邻域构成它的邻域基 \\
            \item X的开子集也是局部紧致的
        \end{enumerate}
    \end{corollary}
    拓扑空间X的覆盖\(\mathscr{U}\)称为局部有限的,如果X的每一点有邻域V,它只同\(\mathscr{U}\)中有限个成员相交.
    设\(\mathscr{U}^{'}\)和\(\mathscr{U}\)都是X的覆盖,如果\(\mathscr{U}^{'}\)的每个成员都包含\(\mathscr{U}\)的某个成员中,则称\(\mathscr{U}^{'}\)是\(\mathscr{U}\)的加细,如果\(\mathscr{U}^{'}\)是开覆盖,则称\(\mathscr{U}^{'}\)为\(\mathscr{U}\)的开加细.
    \begin{definition}
        拓扑空间X称为仿紧的,如果X的每个开覆盖都有局部有限的开加细.
    \end{definition}
    \begin{theorem}
        \begin{enumerate}
            \item 紧致空间是仿紧的 \\
            \item 仿紧的Hausdorff空间满足\(T_4\)公理 \\
            \item 局部仿紧,并满足\(C_2\)公理的Hausdorff空间是仿紧的,从而\(E^n\)是仿紧的 \\
            \item 度量空间是仿紧空间 
        \end{enumerate}
    \end{theorem}
\subsection*{常见性质}
\begin{theorem}
    紧致空间的无穷子集必有聚点
\end{theorem}
\begin{theorem}
    紧致度量空间是可分空间,也是\(C_2\)空间
\end{theorem}
\begin{theorem}
    有限个紧致子集之并集紧致
\end{theorem}
\begin{theorem}
    设A是度量空间(X,d) 的紧致子集,则 \begin{enumerate}
        \item 规定A的直径\(D(A) = \sup \left\{d(x,y) | x,y \in A \right\}\),则存在\(x,y \in A \)使得\(d(x,y) =D(A)\) ;\\
        \item 若\(x \notin A \),则存在\(y \in A \),使得\(d(x,y) =d(x,A)\) ; \\
        \item 若B是X的闭集,\(A \cap B = \emptyset \),则\(d(A,B) \neq 0\)
    \end{enumerate}
\end{theorem}
\begin{theorem}
    如果X的每个紧致子集都是闭集,则X的每个序列不会有两个或者两个以上的极限点 . 
\end{theorem}
\begin{theorem}
    如果\(X \times Y \)是紧致空间,那么\(X ,Y \)也是紧致空间 
\end{theorem}
\begin{theorem}
    X的子集族 \(\mathscr{A}\)称为有核的,如果\(\mathscr{A}\)中任何有限个成员之交非空,则X是紧致空间\(\Leftrightarrow\)X的任何有核闭集族\(\mathscr{A}\) 之交\(\bigcap\limits_{A \in \mathscr{A}} \notin \emptyset\)
\end{theorem}
\begin{theorem}
    度量空间X紧致的充分必要条件是X上任一连续函数都是有界的
\end{theorem}
\begin{theorem}
    局部紧致空间的闭子集也是局部紧致的
\end{theorem}
\begin{theorem}
    如果X满足\(T_3\)公理,则
    \begin{enumerate}
        \item A是X的紧致子集,U是A的邻域,则存在A的邻域V,使得\(\overline{V} \subset U \) \\
        \item X中紧致子集的闭包也紧致
    \end{enumerate}
\end{theorem}
\begin{theorem}
    \(E^n\)一点紧致化同胚于\(S^n\)
\end{theorem}
\section{连通性}
“对图形连通性的认识必须深化.现在,我们要把连通性作为拓扑概念给出严格的定义.直观上的连通,可以有两种含义:其一是图形不能分割成互不“粘连”的两部分;其二是图形上任何两点可以用图形上的线连结.在拓扑学中,这两种含义分别抽象成“连通性”和“道路连通性”两个概念.它们分别在本节和下一节中讨论.这是两个不同的概念.例如对于上面给出的空间X,将看到它连通,但并不道路连通.”“道路连通是在直观连通概念基础上演化来的另一个拓扑性质.对于它,“道路”是关键概念."
\subsection*{连通性定义}
\begin{definition}
    如果一个拓扑空间X不能分解为两个不相交的开集的并,则称拓扑空间X是连通的
\end{definition}
\begin{corollary}
    若拓扑空间X是连通空间,则下述命题等价: 
    \begin{enumerate}
        \item X不能分解为两个非空不相交闭集的并 \\
        \item X没有既开又闭的非空真子集 \\
        \item X的既开又闭的子集只有X和\(\emptyset\)
    \end{enumerate}
\end{corollary}
\subsection*{连通空间的性质}
\begin{corollary}
    连通空间在连续映射下的像也是连通的
\end{corollary}
\begin{proof}
    设X是连通空间,那么X中既开又闭的子集只有X 与\(\emptyset\),又因为连续映射是开(闭)映射,那么因此连通空间的像是连通空间.
\end{proof}
\begin{example}
    \(S^1\)是连通的
\end{example}
\begin{example}
    设\(A \subset E^1\),则 A连通\(\Leftrightarrow\)A是区间
\end{example}
\begin{corollary}
连通空间上的一切连续函数都能取到一切中间值.
\end{corollary}
\begin{proof}
    因为连通空间上连续函数都可以有形式\(f : X \rightarrow E^1\),所以命题成立
\end{proof}
\begin{lemma}
    若\(X_0\)是X的既开又闭的子集,A是X的连通子集,则或者\(A \cap X_0 = \emptyset\),或者\(A \subset X_0\)
\end{lemma}
\begin{proof}
    \(A \cap X_0\)是A的既开又闭的子集,由于A连通,则或者\(A \cap X_0 = \emptyset\),或者\(A \subset X_0\)
\end{proof}
\begin{corollary}
    若X有一个连通的稠密子集,则X连通
\end{corollary}
\begin{proof}
    易证,略
\end{proof}
\begin{corollary}
    若A是X的连通子集,\(A \subset Y \subset \overline{A}\),则Y连通.
\end{corollary}
下面的命题十分重要,它是判断连通空间的一个常用法则
\begin{corollary}
    如果X有一个连通覆盖\(\mathscr{U}\)(\(\mathscr{U}\)中每个成员都连通),并且X有一个连通子集A,它与\(\mathscr{U}\)中每一个成员都相交,则X连通
\end{corollary}
\begin{proof}
    设\(X_0\)是X的既开又闭的子集,往证\(X_0 = \emptyset \text{或} X \), 根据引理可知: \(A \cap X_0 = \emptyset \)或者\(A \subset X_0\).如果\(A \cap X_0 = \emptyset\),那么对于\(\forall U = \in  \mathscr{U}\),因为\(A \cap U \neq \emptyset\),因此\(U \notin X_0\),因此由于引理可知 \(U \cap X_0 = \emptyset\).则 \(X_0 = \emptyset\). 如果\(A \subset X_0\),则\(\forall U \in \mathscr{U}\),\(U \cap X_0 \neq \emptyset\),则\(X = \bigcup\limits_{U \in \mathscr{U}} U \)则\(X_0 =X \)
\end{proof}
\begin{theorem}
    连通性是可乘的
\end{theorem}
\subsection*{连通分支}
连通分支是研究不连通空间时引出的一个概念
\begin{definition}
    拓扑空间X的一个子集称为X的连通分支,如果它是连通的,并且不是X的其他连通子集的真子集,当A是X的连通分支时,如果X的子集\(A \subset B \),并且\(A \neq B \),则B不连通,因此可以说连通分支就是极大连通子集
\end{definition}
\begin{note}
    当X连通时,它只有一个连通分支,就是X自身
\end{note}
\begin{corollary}
    X的每个非空连通子集包含在唯一的一个连通分支中
\end{corollary}
\begin{corollary}
    连通分支是闭集
\end{corollary}
\begin{proof}
    设A是X的一个连通分支,所以\(\overline{A}\),并且由于\(A =\overline{A}\),所以A是闭集
\end{proof}
\subsection*{局部连通性}
\begin{definition}
    拓扑空间X称为局部连通的,如果\(\forall x \in X \),x 的所有连通邻域构成x的邻域基
\end{definition}
按照定义,当X局部连通时,如果U是点x的邻域,则必有x的连通邻域\(V \subset U \)
\begin{note}
    注意:连通空间不一定是局部连通空间
\end{note}
\begin{corollary}
    局部连通空间的连通分支是开集
\end{corollary}
\subsection*{道路连通性}
首先回顾一下道路的定义
\begin{definition}
    设X是拓扑空间,从单位闭区间\(l= [0,1]\)到X的一个连续映射\(a: l \rightarrow X\)称为X上的一个道路,把点\(a(0)\)和\(a(1)\)分别称为a的起点和终点,统称为端点 
\end{definition}
道路是指映射本身,而不是它的像集.事实上可能有许多不同道路,它们的像集完全相同.
\\
如果道路\(a : | \rightarrow X\)是常值映射,即\(a(I)\)是一个点,就称为点道路,点道路完全被像点x决定,记为\(e_x\).起点和终点重合的道路称为闭路,
道路有两种运算: 逆和乘积 
\begin{definition}
    一条道路\(a : I \rightarrow X \)的逆也是X上道路,记作\(\overline{a}\),规定为\(\overline{a}(t)=a(1-t),\forall t \in I\)
\end{definition}
X上的两条道路a与b,如果满足\(a(1) = b(0)\),则可以规定它们的乘积ab,它也是X上的道路,规定为
\begin{align}
    ab(t)= \begin{cases}
        a(2t) \qquad & 0 \leq t \leq \frac{1}{2} \\ 
        b(2t-1) \qquad & \frac{1}{2}\leq t \leq 1
    \end{cases}
\end{align}
下面列出关于逆和乘积的几个性质:
\begin{theorem}
    \begin{enumerate}
        \item \(\overline{e_x} = e_x\) \\
        \item \(\overline{(\overline{a})}=a\) \\
        \item 当ab有意义时,\(\overline{b}\overline{a}\)有意义,且\(\overline{b}\overline{a}=\overline{ab}\)
    \end{enumerate}
\end{theorem}
需要注意的是,道路的概念十分重要,它是建立基本群的基础
\subsection*{道路连通空间}
\begin{definition}
    拓扑空间X称为道路连通的,如果\(\forall x,y \in X \),存在X中分别以x和y为起点和终点的道路
\end{definition}
\begin{example}
    \(E^n\)是道路连通的,一般地若A是\(E^n\)中的凸集,则A是道路连通的
\end{example}
\begin{corollary}
    道路连通空间一定是连通空间
\end{corollary}
\begin{proof}
    设X是道路连通空间,\(\forall x \in X \),都有一点与x形成道路,所以X只有一个连通分支,因此X是连通空间
\end{proof}
\begin{corollary}
    道路连通空间的连续映像是道路连通的.
\end{corollary}
\begin{proof}
    同连通空间证法,略
\end{proof}
\begin{theorem}
    道路连通性是可乘的
\end{theorem}
\subsection*{道路连通分支}
在拓扑空间X中,规定它的点之间的一个关系\(\sim\):若点x与y可用X上的道路连结,则说x与y相关,记作\(x \sim y\),这是一个等价关系: \(e_x\)连结x与自己,有自反性;当a连结x与y,\(\overline{a}\)连结y与x,有对称性;如果\(x \sim y ,y \sim z \)则\(x \sim z\)
得到传递性
\begin{definition}
    拓扑空间在等价关系\(\sim\)下分成的等价类称为X的道路连通分支,简称道路分支
\end{definition}
按照定义,\(\forall x \in X \)属于X的唯一道路分支;X的每个道路连通的自己包含在某个道路分支中;X道路连通的充分必要条件是它只有一个道路分支 
\begin{corollary}
    拓扑空间的道路分支时它的极大道路连通子集.
\end{corollary}
“从命题立即可推出,X的每个道路分支都连通,因此必包含在某个连通分支中.于是,X的每个连通分支是一些道路分支的并集.”
\subsection*{局部道路连通}
\begin{definition}
    拓扑空间X称为局部道路连通的,如果\(\forall x \in X \),x的道路连通邻域构成x的邻域基.
\end{definition}
\begin{note}
    与局部连通类似,局部道路连通空间不一定是道路连通空间
\end{note}
\begin{lemma}
       如果拓扑空间X的每一个点x有邻域\(U_x\),使得x与\(U_x\)中每一个点都可用X上道路连结,则
       \begin{enumerate}
           \item X的道路分支都是既开又闭的; \\
           \item X的连通分支就是道路分支
       \end{enumerate}
\end{lemma}
\begin{theorem}
    局部道路连通空间X的道路分支就是连通分支.它们是既开又闭的 ; 当X连通时;它一定道路连通.
\end{theorem}
\subsection*{常见性质}
\begin{theorem}
    平凡拓扑空间连通;包括两个以上点的离散拓扑空间不连通.
\end{theorem}
\begin{theorem}
    设\(X_1,X_2\)都是连通空间X的开子集,\(X_1 \cup X_2 = X\),\(X_1 \cap X_2\)非空,并且连通,则\(X_1,X_2\)是连通空间.
\end{theorem}
\begin{theorem}
    设X是满足\(T_1.T_4\)公理的连通空间,并且X中至少有两个点,则X是不可数的
\end{theorem}
\begin{theorem}
    局部连通空间的开子集也局部连通
\end{theorem}
\begin{theorem}
    X不连通 \(\Leftrightarrow \)存在定义在X上的连续函数\(f : X \rightarrow E^1\),使得f(x)是两个点
\end{theorem}
\begin{theorem}
    设X是紧致Hausdorff空间,\(\mathscr{F}\)是X的一族连通闭子集,并且\(\mathscr{F}\)中任何有限个成员之交是非空连通集,则\(\bigcap\limits_{F \in \mathscr{F}} F\)是非空的连通集
\end{theorem}
\begin{theorem}
    设\(A \subset E^2\),\(A^{c}\)是可数集,则A是道路连通空间
\end{theorem}
\begin{theorem}
    如果\(X_1 ,X_2\)都是X的开集,\(X= X_1 \cup X_2 = X \),a是X上的道路,它的两个端点分别在\(X_1 , X_2\)中,则\(a^{-1}(X_1 \cap X_2)\)非空.   
\end{theorem}
\begin{theorem}
    如果\(X_1 ,X_2\)都是X的开集,\(X = X_1 \cup X_2\),并且X与\(X_1 \cap X_2\)都道路连通,则\(X_1 , X_2\)都是道路连通空间
\end{theorem}
\begin{theorem}
    \(E^1\)与\(E^n\)不同胚
\end{theorem}
\begin{theorem}
    两条相交直线的并集与一条直线不同胚
\end{theorem}
