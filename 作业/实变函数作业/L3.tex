\chapter{测度理论} % (fold)
\label{chap:测度理论}
\section{开集的体积} % (fold)
\label{sec:开集的体积}
1.对\(\mathbb{R}^n\)中点\(x = (x_1,x_2 ,\dots ,x_n),y=(y_1,y_2,\dots,y_n)\),记\(x+y = (x_1 + y_1 ,x_2+y_2,\dots ,x_n+y_n)\),设G
是\(\mathbb{R}^n\)中开集,\(x\in \mathbb{R}^n \).令\(\widetilde{G}=\left\{x+y ; y\in G\right\}\),那没\(\widetilde{G}\)也是开集,证明
\(|G|=|\widetilde{G}|\)
\begin{proof}[Proof ]
  根据第二章第四节的定理可知: \begin{align*}
    G = \bigcup\limits_{i=1}^{\infty} G_i 
  \end{align*} 
  其中\(G_i\)为不交的左闭右开的区间,则令\(\widetilde{G}_i =\left\{x+y ; x \in G_i \right\} \)
  同时因为\(|G_i| =|\widetilde{G}_i| \)
  所以
  \begin{align*}
    |\widetilde{G}| &= |\bigcup\limits_{i=1}^{\infty} \widetilde{G_i}| \\
    &= |\bigcup\limits_{i=1}^{\infty} G_i| \\
    &=|G|
  \end{align*}
  得证.
\end{proof}
2.设\(I\)是\(\mathbb{R}^2\)中的一个开区间,G是I绕点旋转\(\frac{\pi}{6}\)后得到的集合,那么G是\(\mathbb{R}^2 \)中开集.证明:\(|G|=|I|\)
\begin{proof}[Proof ]
 空 
\end{proof}
3.设G是\(\mathbb{R}^n \)中开集,令
\begin{align*}
  m^{*}G &= \inf\left\{\sum_{i=1}^{\infty}|I_i| , I_i (i=1,2,3,\dots ) \text{是开区间,且} \bigcup\limits_{i=1}^{\infty} I_j \subset G \right\}
  \\ 
  m_{*}G &= \sup\left\{\sum_{j=1}^{\infty}|F_j| ,\text{k是正整数,} F_j (j=1,2,\dots , k) \text{是互不相交的含于G的闭区间}\right\}
\end{align*}
证明: \(m^{*}G =m, G=|G| \)
\begin{proof}[Proof ]
 空 
\end{proof}
4.设G是\(\mathbb{R}^2 \)中开集,\(\widetilde{G}\) 是G绕原点旋转\(\theta\)后得到的集合易见\(\widetilde{G}\)也是开集,
证明\(|G| = |\widetilde{G}| \)
\begin{proof}[Proof ]
  空  
\end{proof}
\section{点集的外测度} % (fold)
\label{sec:点集的外测度}

% section 点集的外测度 (end)
% section 开集的体积 (end)
% chapter 测度理论 (end)
