\section{第二周}
\subsection{聚点,内点,边界点,Bolzano-Weirtdtrass定理}
先介绍基本的点集理论,对于\(\mathbb{R}^n\)来说,指的是n个实数组成的数组称为\(\mathbb{R}^n\)中的点,给出\(\mathbb{R}^n\)中的\(X = \bigl(x_1,x_2,\dots,x_n \bigr)\),\(Y= \bigl(y_1,y_2,\dots,y_n\bigr)\)
之间的距离\(\rho(X,Y)\)定义为
\begin{align*}
    \rho(X,Y) = \left\{\sum_{i=1}^{n}(x_i^2 - y_i^2)\right\}^{\frac{1}{2}}
\end{align*}
\begin{Theorem}
    存在如下不等式:
    \begin{align*}
        \rho(X, Y) + \rho(Y,Z) > \rho(X,Z) \\ 
        \rho(X,Y) - \rho(Y,Z) < \rho(X,Z)
    \end{align*}
\end{Theorem}
记\[N(X, \delta)=\left\{Y; \rho(X,Y) < \delta \right\}\]
称上述的运算为以X为原点,\(\delta\)为半径的邻域.
对于\(\mathbb{R}^n\)中的点集M来说,如果\(\forall X = \bigl(x_1,x_2 ,\dots,x_n\bigr) \in M \)均有\(|x_i| < K \quad i=1,2,3,\dots ,n\)
则称集合M是有界的.
\subsection{聚点,内点,边界点}
考虑\(X \in \mathbb{R}^{n}\)和\(E \subset \mathbb{R}^n\)之间的关系,现在有三种可能: \\ 
1. 对于\(\exists \delta > 0 \)都有: \(N(X , \delta) \subset E\) \\ 
2. 对于\(\forall \delta > 0\)都有 : \(N(X ,\delta) \bigcap E \neq \emptyset\)并且 \(N(X, \delta) \bigcap E^{c} \neq \emptyset\) \\
3.对于\(\exists \delta > 0 \)都有\(N(X,\delta) \subset E^{c}\) \\
我么称1中所述的点为集合E的内点,2中所述的点为集合E的边界点,3中所述的点为集合E的外点 . 
\begin{Definition}
    设\(E \subset \mathbb{R}^{n} , P_0 \in \mathbb{R}^n\),如果对于任意的\(\delta > 0\),在以\(P_0\)为心,以\(\delta\)为半径的邻域\(N (P_0 , \delta)\)中恒有无穷多个点属于E,则我们称\(P_0\)是E的聚点.
\end{Definition}
显而易见的,所有内点均是聚点.E的聚点不一定属于E
\begin{Theorem}
    \(P_0 \in E^{'}\)的充分必要条件是\(P_0\)为E之一极限点,即有一串互异的点\(P_n \in E\),使得\(\rho (P_0 , P_n) \rightarrow 0 \quad(n \rightarrow \infty )\)
\end{Theorem}
\begin{proof}
    充分性:若\(\rho (P_0 , P_n) \rightarrow 0 \quad(n \rightarrow \infty )\),那么\(\forall \delta > 0 \),\(\rho(P_n - P_o) < \delta\) 根据定义\(P_n \in E^{'}\)
    \\ 
    必要性:若\(p_0 \in E^{'}\)从而\(P_0\)是E的聚点,那么\(\forall \delta > 0 \) , \(N(P_0 , \delta ) \)中恒有无穷多个点属于E,那么取\(\delta = \frac{1}{n}\),从而得到一序列\(\left\{P_1,P_2,\dots,P_n\right\}\)
\end{proof}
\begin{Theorem}
    若\(A \subset B \subset \mathbb{R}^n\),则\(A^{'} \subset B^{'}\)
\end{Theorem}
\begin{Theorem}
    若\(A \subset \mathbb{R}^{n}  ; B \subset \mathbb{R}^n \)从而\((A \cup B)^{'} \subset A^{'} \cup B^{'}\)
\end{Theorem}
是E的边界点而不是E的聚点的点称为E的孤立点,显然,E中孤立点一定属于E,如果集合E的每一个点都是孤立点,则称E为孤立集合
\begin{Theorem}
    E是孤立集合的充要条件为\(E \bigcap E^{'} = \emptyset\)
\end{Theorem}
\subsection{Bolzano–Weierstrass定理}
若\(A \subset R^{n}\)是一有界无穷集,那么A中至少有一个聚点,即\(A^{'} \neq \emptyset\)
\subsubsection*{画格法}
考虑如下集合\[\left\{(x,y) ; a_1 < x <a_2 , b_1<y<b_2 \right\}\]不失一般性
从坐标系系统来看: 它是以原点为中心的长方形(如图\ref{画格长方形})以坐标轴为割线,将该长方形分为四个小长方形,取右上长方形,重复操作,我们可以根据\(\textbf{Cantor}\)闭区间套定理可知,一定能取到A的聚点.
\begin{figure}[h]
    \centering
    \includegraphics[width=0.5\linewidth]{figures/image_4.png}
    \caption{画格长方形}
    \label{画格长方形}
\end{figure}
\begin{note}
    画格法在分析学中常常使用并且阿尔弗斯在它的教材中也常常使用.
\end{note}
\begin{Definition}
    设\(E \subset \mathbb{R}^{n} \),称E的全体聚点所作成的点集为E的导集,记为\(E^{'}\)又\(\overline{E}=E \bigcup E^{'}\) 
\end{Definition}

\subsection{开集,闭集与完备集}
\begin{Theorem}
    设\(\left(x,\tau\right)\),那么闭集有性质:
    \begin{enumerate}
        \item \(\emptyset,X\)闭 \\
        \item 闭集任意交封闭 \\
        \item 闭集有限并封闭
    \end{enumerate}
\end{Theorem}
\begin{note}
    上述定义利用:公式:
    \begin{enumerate}
        \item \({(\bigcap A_{\lambda})}^{c}=\bigcup(A_{\lambda}^{c})\) \\
        \item \({(\bigcup A_{\lambda})}^{c}=\bigcap(A_{\lambda}^{c})\)
    \end{enumerate}
\end{note}
\begin{Definition}
    对于拓扑空间\(x,\tau\),\(a \subseteq X,\tau \in A\),如果存在一个开集U,s.t \(x \in U \subseteq A\),则称x是A的一个内点,A是X的一个邻域。
    A的所有内点称为A的内部,记作\(A^{o}\)
\end{Definition}
\begin{Corollary}
    \begin{enumerate}
        \item 若\(A \subseteq B\),则\(A^{o}\subseteq B^{o}\) \\
        \item \(A^{o}\)是包含于A的最大的开集\\
        \item A是开集 \(\Longleftrightarrow \) \(A=A^{o}\) \\
        \item \({\left(A \cap B\right)}^{o} = {A}^{o}\cap {B}^{o}\)\\
        \item \({\left(A \cup B\right)}^{o} \supseteq A^{o}\cup B^{o}\)
    \end{enumerate}
\end{Corollary}
\begin{proof}
    证明命题1: 对于\(\forall x\in A^{o}\) 存在开集u ,s.t \(x \in u \subseteq A \subseteq B\),则 \(x \in \subset B^{o}\)
    即\(A^{o} \subset b^{o}\) \\
    证明命题2: 记\(\mathbf{U}=\left\{u,\text{u是开集},u\subset A\right\}\),命题等于证 \[\bigcup_{u \in \mathbf{U}}u=A^{o}\]
    可以使用”两边夹“方法: \\
    先证明: \(\bigcup_{u \in \mathbf{U}}u \supseteq A^{o}\) : 任取\(x_0 \in A^{o}\),根据定义有 \[\exists u_i , x_O \in U_i \subseteq A \]
    即\(u_i \in \mathbf{U}\),因而 有:\[x_0 \in u_i \subseteq \bigcup_{u \in \mathbf{U}}u\]
    再证明: 只需要每个\(u_i \in \mathbf{U}\),有 \(u_i \subseteq A^{o}\) 对 \(\forall x \in u_i\),则
    \[u_i \subseteq A \quad \text{则} x_0 \in A ^{o}\] \\
    证明命题3: 命题3是命题2的推论,可以直接得到. \\
    证明命题4:1.\({\left(A \cap B\right)}^{o}\subseteq A^{o} \quad {\left(A \cap B\right)}^{o}\subseteq B^{o}\)从而有 
    \[{\left(A \cap B\right)}^{o}\subseteq A^{o}\cap B^{o}\]
    2.\(A^{o}\cap B^{o} \subseteq A \cap B\)和\({\left(A^{o}\cap B^{o}\right)}^{o}\subseteq {\left(A\cap B\right)}^{o}\)
    从而命题4得证
    证明命题5:方法同命题4类似,略.
\end{proof}
\begin{note}[命题四]
    \begin{enumerate}
        \item \({\left(\bigcap_{k=1}^{n} A_k\right)}^{o}=\bigcap_{k=1}^{n}{A_k}^{o}\)\\
        \item \({\left(\bigcap_{k=1}^{\infty} A_k\right)}^{o} \subseteq \bigcap_{k=1}^{\infty}{A_k}^{o}\)
    \end{enumerate}
\end{note}
\begin{note}
    何为最大性,它是最大的,比任何具有相同性质的都大。 那么对于开集的最大性就有: \\
    指任意包含于A的开集\(\varpi\) 有
    \[\varpi \subseteq A^{o} =\bigcup_{u \in \mathbf{U}} u\]
    这是接触到的第一个新事物,在刚开始学习拓扑学中,应该一步步地严谨证明。
\end{note}
\begin{Definition}
    \( \left(x,\tau\right) \quad A \subseteq X \quad x \in X\) 如果x的任意邻域都包含\(A \setminus \left\{x\right\}\)的点,则称x是A的一个聚点,A的聚点的全体组成的集合叫做A的导集\(A^{'}\),记A的闭包为\(\overline{A}=A \cup A^{'}\)
\end{Definition}
\(x \in \overline{A} \Leftrightarrow x \in A 或 x\in A^{'} \Leftrightarrow x的任意邻域与A交非空\)
\begin{Definition}[边界点]
    \(\partial\),x是A的边界点,如果x的任一邻域,既与A交非空,又与\(A^{c}\)交非空.
\end{Definition}
\begin{proof}
    \({\partial A }^c =A^o \cup {\left(A^c\right)}^o\)
\end{proof}
\subsection{闭集}
\begin{Corollary}
    \begin{enumerate}
        \item 若\(A \subseteq B\),则 \(\overline{A} \subseteq \overline{B}\) \\
        \item \(\overline{A}\)是包含A的最小的闭集。\(\left(\text{等价于: 所有包含A的闭集的交}\right)\) \\
        \item \(A = \overline{A} \Longleftrightarrow \text{A是闭集} \) \\
        \item \(\overline{A \cup B} = \overline{A} \cup \overline{B}\) \\
        \item \(\overline{A \cap B} \subseteq \overline{A} \cap \overline{B}\)
    \end{enumerate}
\end{Corollary}
\begin{proof}
    证明方法同开集性质证明方法.
\end{proof}
\begin{Corollary}
    \(\left(x,\tau\right)\),\(A \subseteq X\),则\[{\overline{A}}^{c}={\left(A^c\right)^{o}} \qquad {A^o}^c=\overline{A^c}\]
\end{Corollary}
\begin{proof}
    由\(x \in \mathbb{C}\)可知,存在x的一个邻域u使得 \[u \in A^c \Longleftrightarrow x \in {A^c}^o\]
\end{proof}
\begin{note}
    上述的证明实际上没有难度,重要的是掌握这种方法,并熟练地运用和准确地写出这些过程
\end{note}
\begin{example}
    \(X=\left\{a,b,c\right\} \quad \tau=\left\{\emptyset,X,\left\{a\right\}\right\}\)
    ,\(A = \left\{a\right\} \quad B=\left\{b\right\}\)
\\
    那么 \[A^o=\left\{a\right\},\overline{A}=X,A^{'}=\left\{b,c\right\},\partial A =\left\{b,c\right\}\]
\end{example}
\begin{example}[Cantor三分集]
    将闭区间\([0,1]\)分成三份,删去中间的开区间\(\bigl(\frac{1}{3}, \frac{2}{3}\bigr)\) 留下两个闭区间\[[\frac{2}{3},1],[0,\frac{1}{3}]\],又把这两部分都分为三段,删去中间的开区间,即\((\frac{1}{9},\frac{2}{9}),(\frac{7}{9},\frac{8}{9})\),如此作下去,我们自然可知:有些点是永远删不下去的.
    我们将所有永远删不去的点所作成的集合E为Cantor集合,Cantor三分集有如下的性质
    \begin{enumerate}
        \item E是闭集 \\ 
        \item E是自密的 ,即\(E = E^{'}\) \\ 
        \item \(\overline{\overline{E}} =c\) 
    \end{enumerate}
\end{example}
\section{p进位表数法}
\begin{Theorem}[二进制的等价性]
    设\(\mathscr{D}\)代表所有由\(0,1\)两个数字重复排列而成的序列;则\(\mathscr{D} \sim (0,1)\)
\end{Theorem}
\begin{proof}[Proof ]
    如果我们将\((0,1)\)中的任意x用二进制小数标出,则确实每一个x都对应由0,1重复排列的序列.但是这种对应不是一一对应的,因为每一个形如\(\frac{m}{2^n}\)的数都有两种表述法,因此对应两个这样的序列. 现在我们约定只用第一种表示法,于\(\forall x \in (0,1)\)就对应于唯一的一个由0,1重复排列的序列,现在我们将对应于删去了的二进制表示法的那些序列以及序列\(0,0,0,\dots\)和\(1,1,1,\dots\)作成集合T,则T自然是可数的,而\(\mathscr{D}- T \)与
    \((0,1)\)的对应已经是一一对应了,所以
    \[\mathscr{D} = (\mathscr{D}-T)\bigcup T \sim (0,1)\bigcup N  \]
    因此:\(\mathscr{D} \sim (0,1)\)
\end{proof}
\section{一维开集,闭集,完备集的构造}
\begin{Theorem}[\(\mathbb{R}^1\)中有界开集构造原理]
    任何非空的有界开集都是有限多个或者是可数多个互不相交的开区间的并,这些开区间的端点都不属于这个开集
\end{Theorem}
\begin{proof}[Proof ]
    设开集\(G \subset (-M,M)\),对于\(\forall x \in G\), 因为G是开集,所以有开区间\(\alpha ,\beta \),使得\(x \in (\alpha ,\beta ) \subset G\).
    利用片段\(\textbf{Segement}\)
    \begin{align}
        E_{\alpha }= \left\{x ; x\notin F , x \leq \alpha \right\} \\ 
    \end{align}
    显然\(E_{\alpha }\)是非空的且以\(\alpha \)为一上界,记\(E_{\alpha }\)的上确界为\(\alpha^{'}\)则\(\alpha^{'} \leq \alpha \),而且\(\alpha^{'} \notin G\) ,因为若\(\alpha^{‘} \in G\),则\(\epsilon > 0 \)充分小时,\((\alpha^{'} - \epsilon ,\alpha^{'} +\epsilon) \subseteq G\),而且\(\alpha^{'}\)是\(E_{\alpha}\)的上确界,所以应有\(y \in E_{\alpha}\),使得\(\alpha^{'} \geq y> \alpha^{'} -\epsilon \),由\(E_{\alpha}\)定义,\(y\notin G\)
    与\((\alpha^{'}-\epsilon ,\alpha^{'}+\epsilon ) \subset G\)矛盾,可见\(\alpha^{'} \notin G\)必然成立,但在\((\alpha ,\alpha^{'})\)之上是不能有点属于G的,故\((\alpha^{'},\beta^{'}) \subset G\)
    类似地,可以适当放到\(\beta\)或者\(\beta^{'}\),使得\((\alpha^{'},\beta^{'}) \subset G\),\(\alpha^{'},\beta^{'} \notin G\),这事实上是说将\(\alpha,\beta\)尽可能地放大,直到遇到不属于G的点为止,如果以\(I_{x}\)表示这样得出来的包含x的开区间\((\alpha^{'},\beta^{'})\).则显然,对于不同的点\(x,x^{'} \in G\),或者\(I_x = I_y\)或者\(I_{x} \bigcap I_{x^{'}} = \emptyset\).因此\(\left\{I_{x}\right\}_{x\in G}\)中至多有可数个彼此互异的开区间.设其为
    \begin{align*}
        I_1 ,I_2,I_3,\dots
    \end{align*}
    则显然\(G \subset \bigcup\limits_{i=1}^{\infty}\)(\(m_i\)有限或者\(\infty\)),但另一方面\(G \subset I_{x}\)恒成立,故又有\(G \subset \bigcup\limits_{i=1}^{\infty} I_i\),于是\(G =\bigcup\limits_{i=1}^{\infty} I_i\)(\(m_i\)有限或者\(\infty\))
\end{proof}
\begin{Theorem}
    设F是一非空的有界闭集,则F中必有一最大点(最大数)和一最小点(最小数)
\end{Theorem}
\begin{Theorem}
    设F是一非空的有界闭集,则F是由一闭区间中去掉有限个或者可数多个互不相交的开区间而成的,这些开区间的端点都还是属于F的
\end{Theorem}
上述定理中所述的各个区间,通常称为F的\(|textbf{邻接区间}\)
\begin{Theorem}
    非空的有界闭集F是完备集合的充要条件是:F是从一闭区间\([a,b]\)中去掉有限个或者可数多个彼此没有公共端点且与原来的闭区间也没有公共端点的开区间而成,这些区间的端点都是属于F的
\end{Theorem}
\section{点集之间的距离}
\begin{Theorem}
    设E是一点集,\(d>0\),U是所以到E的距离小于d的点点P作成的点集,即\begin{align*}
        U = \left\{P ; \rho(P,E) < d\right\}
    \end{align*}
    则U是一开集,且\(U \supset E\)
\end{Theorem}
\begin{Theorem}[隔离性定理]
    设\(F_1,F_2\)是两个非空有界闭集,\(F_1 \bigcap F_2 = \emptyset\),则有开集\(G_1,G_2\),使\(G_1 \subset F_1\),\(G_2 \subset F_2\),\(G_1 \bigcap G_2 = \emptyset\)
\end{Theorem}

